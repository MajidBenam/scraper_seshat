\documentclass[14pt,a4paper]{extarticle}
\usepackage[german]{babel}
\usepackage[letterpaper,top=1.5cm,bottom=1.5cm,left=1.5cm,right=1.5cm,marginparwidth=1.5cm]{geometry}
\begin{document}
\section{Text mit Blanks}Wer (---) siebten Himmel schwebt, verliert oft (---) Boden unter (---) Füßen und muss (---) anderen zurück (---) (---) Teppich geholt werden. (---) nah und fern, oben und unten dreht (---) vieles (---) (---) deutschen Sprache.\newline\newline \newline\newline Aus nahe liegenden Gründen lehnte (---) junge Verkäuferin (---) ab, (---) alten Firmenchef (---) heiraten. (---) war nämlich abgrundtief hässlich. Außerdem war (---) (---) (---) beide Ohren (---) (---) jungen Kollegen verliebt, seit Wochen schon „im siebten Himmel“, weil (---) (---) auch mochte. Noch nie hatte (---) (---) jemandem (---) nah gefühlt, und Geldgier lag (---) ohnehin fern. (---) (---) (---) Angebot (---) alten Herrn hörte, war (---) (---) allen Wolken gefallen. Ganz rot war (---) geworden und vor Überraschung regelrecht verstummt. Klar fühlte (---) (---) auch geehrt. Aber nur (---) finanziellen Gründen (---) Mann (---) heiraten, nein, (---) tief wollte (---) nicht sinken.  (---) beiden Beinen fest (---) (---) Boden  Bodenständige Politiker kommen bei (---) meisten Bürgern gut an, Menschen, (---) Sinn für Alltagsprobleme haben, und (---) auch (---) einfachen Worten ausdrücken können. (---) ziehen (---) Sympathie (---) sich, ganz (---) Gegenteil (---) denen, (---) „hochtrabend“, angeberisch, auftreten, weitschweifige Erklärungen abgeben und abgehoben, abstrakt, daherreden. Wer also Erfolg bei (---) Wählern haben will, muss zumindest (---) wirken, (---) stehe er (---) beiden Beinen fest (---) (---) Boden (---) Tatsachen.  Andererseits gibt (---) immer wieder sprachgewandte Ideologen, (---) Unerfahrene (---) hochfliegenden Ideen begeistern, sowie (---) immer noch junge Frauen gibt, (---) (---) charmante Männer hereinfallen, (---) ihnen versprechen, für (---) (---) Sterne (---) Himmel (---) holen.  Wer hoch hinaus will, muss Tauben fangen können  Erfahrene Menschen hingegen wissen, dass (---) (---) (---) Nähe betrachtet viele schöne Ideen und Versprechungen rasch (---) Luft auflösen, weil (---) einfach nicht (---) verwirklichen sind. Hier könnten jetzt Psychologen ins Spiel kommen und betonen, dass manchmal hochgesteckte Ziele notwendig sind, (---) (---) (---) (und (---) Fähigkeiten) hinauszuwachsen.  Allen Erwartungen (---) Trotz wird (---) (---) Bankangestellte vielleicht (---) erfolgreicher Schauspieler und (---) Bäcker (---) Schriftsteller berühmt. Doch (---) Pragmatiker predigen seit jeher (---) Sicherheit. (---) geben (---) lieber (---) weniger zufrieden (---) Risiken einzugehen. (---) Lieblingsspruch lautet: „Besser (---) Spatz (---) (---) Hand (---) (---) Taube (---) (---) Dach.“  Weltfremd oder hoffnungsfern  (---) (in gewisser Weise) fern (---) Welt gilt (---) besondere Sorte kluger Menschen. Nämlich jene, (---) (---) Elfenbeinturm leben, zumeist Wissenschaftler (---) Spezialgebieten, (---) abgeschieden und unberührt (---) Rest (---) Gesellschaft (---) Dasein verbringen und forschen, (---) Beispiel (---) (---) Liebesleben (---) Nacktschnecken oder grammatische Sonderentwicklungen lateinischer Hilfsverben (---) Mittelalter. Ihre Vorträge füllen selten große Säle (---) Zuhörern. Anders sieht (---) bei Popstars aus: Hier strömen (---) Massen (---) nah und fern, also (---) überall heran.  Wer ganz unten ist, also (---) richtig „down“, (---) nun traurig, weil (---) (---) Geliebte verlassen hat, oder arm, weil er keine Arbeit findet, für (---) ist (---) Glück meist himmelsfern, meilen- oder gar sternenweit weg. (---) jemand braucht (---) kleinen Silberstreif (---) Horizont, (---) kleine ferne Hoffnung (---) Besserung (---) (---) Zukunft. Man muss ja nicht gleich davon träumen, (---) (---) oberen Zehntausend (---) gehören und (---) (---) großen Jachten (---) Kellnern bedienen (---) lassen. Auch (---) (---) Wünschen sollte man schön (---) (---) Teppich bleiben.  (---) (---) Stimmung kommt (---) (---)  (---) Ausblick (---) (---) neue Anstellung, (---) wenig Aufsteigen (---) Beruf, (---) wäre schon etwas - oder (---) versöhnendes Gespräch (---) (---) tief Verehrten. Warum (---) Geliebte nun tief verehrt wird, (---) Zirkusdirektor aber (---) Gäste (---) „hoch verehrtes Publikum“ anredet, darüber klärt uns (---) Sprache leider nicht auf. Positiv aber ist beides.  Ähnlich steht’s (---) (---) Gefühl, nur kann man hier (---) Stimmungen unterscheiden. (---) Hochgefühl geht (---) (---) Freude, Jubel und Heiterkeit, (---) tiefen Gefühl eher (---) Nachdenkliches oder Trauriges.  Hochgefühl (---) Glücks und tiefe Dankbarkeit  (---) Rennfahrer kann (---) Hochgefühl seines Erfolgs zufrieden (---) (---) Kamera lächeln. (---) beinah (---) Seenot Ertrunkene empfindet (---) tiefes Gefühl (---) Dankbarkeit für (---) Retter. (---) Hochgefühl (---) Glücks lädt (---) Lottogewinner alle (---) Bekannten (---) einer großen Party ein, (---) einem tiefen Gefühl (---) Zuneigung blickt (---) Pfarrer (---) (---) Gemeinde. Und wer alle Höhen und Tiefen (---) Lebens und (---) Gefühle schon einmal durchgemacht hat, (---) kann man (---) Recht (---) weise und erfahren bezeichnen.\newline\newline \newline\newline \section{Text Ohne Blanks}Wer im siebten Himmel schwebt, verliert oft den Boden unter den Füßen und muss von anderen zurück auf den Teppich geholt werden. Um nah und fern, oben und unten dreht sich vieles in der deutschen Sprache.\newline\newline \newline\newline Aus nahe liegenden Gründen lehnte die junge Verkäuferin es ab, den alten Firmenchef zu heiraten. Der war nämlich abgrundtief hässlich. Außerdem war sie bis über beide Ohren in einen jungen Kollegen verliebt, seit Wochen schon „im siebten Himmel“, weil der sie auch mochte. Noch nie hatte sie sich jemandem so nah gefühlt, und Geldgier lag ihr ohnehin fern. Als sie vom Angebot des alten Herrn hörte, war sie aus allen Wolken gefallen. Ganz rot war sie geworden und vor Überraschung regelrecht verstummt. Klar fühlte sie sich auch geehrt. Aber nur aus finanziellen Gründen einen Mann zu heiraten, nein, so tief wollte sie nicht sinken.  Mit beiden Beinen fest auf dem Boden  Bodenständige Politiker kommen bei den meisten Bürgern gut an, Menschen, die Sinn für Alltagsprobleme haben, und das auch in einfachen Worten ausdrücken können. Sie ziehen die Sympathie auf sich, ganz im Gegenteil zu denen, die „hochtrabend“, angeberisch, auftreten, weitschweifige Erklärungen abgeben und abgehoben, abstrakt, daherreden. Wer also Erfolg bei den Wählern haben will, muss zumindest so wirken, als stehe er mit beiden Beinen fest auf dem Boden der Tatsachen.  Andererseits gibt es immer wieder sprachgewandte Ideologen, die Unerfahrene mit hochfliegenden Ideen begeistern, sowie es immer noch junge Frauen gibt, die auf charmante Männer hereinfallen, die ihnen versprechen, für sie die Sterne vom Himmel zu holen.  Wer hoch hinaus will, muss Tauben fangen können  Erfahrene Menschen hingegen wissen, dass sich aus der Nähe betrachtet viele schöne Ideen und Versprechungen rasch in Luft auflösen, weil sie einfach nicht zu verwirklichen sind. Hier könnten jetzt Psychologen ins Spiel kommen und betonen, dass manchmal hochgesteckte Ziele notwendig sind, um über sich (und seine Fähigkeiten) hinauszuwachsen.  Allen Erwartungen zum Trotz wird so der Bankangestellte vielleicht ein erfolgreicher Schauspieler und der Bäcker als Schriftsteller berühmt. Doch die Pragmatiker predigen seit jeher die Sicherheit. Sie geben sich lieber mit weniger zufrieden als Risiken einzugehen. Ihr Lieblingsspruch lautet: „Besser der Spatz in der Hand als die Taube auf dem Dach.“  Weltfremd oder hoffnungsfern  Als (in gewisser Weise) fern der Welt gilt eine besondere Sorte kluger Menschen. Nämlich jene, die im Elfenbeinturm leben, zumeist Wissenschaftler mit Spezialgebieten, die abgeschieden und unberührt vom Rest der Gesellschaft ihr Dasein verbringen und forschen, zum Beispiel über das Liebesleben der Nacktschnecken oder grammatische Sonderentwicklungen lateinischer Hilfsverben im Mittelalter. Ihre Vorträge füllen selten große Säle mit Zuhörern. Anders sieht das bei Popstars aus: Hier strömen die Massen von nah und fern, also von überall heran.  Wer ganz unten ist, also so richtig „down“, ob nun traurig, weil ihn die Geliebte verlassen hat, oder arm, weil er keine Arbeit findet, für den ist das Glück meist himmelsfern, meilen- oder gar sternenweit weg. So jemand braucht einen kleinen Silberstreif am Horizont, eine kleine ferne Hoffnung auf Besserung in der Zukunft. Man muss ja nicht gleich davon träumen, zu den oberen Zehntausend zu gehören und sich auf großen Jachten von Kellnern bedienen zu lassen. Auch mit seinen Wünschen sollte man schön auf dem Teppich bleiben.  Auf die Stimmung kommt es an  Der Ausblick auf eine neue Anstellung, ein wenig Aufsteigen im Beruf, das wäre schon etwas - oder ein versöhnendes Gespräch mit der tief Verehrten. Warum die Geliebte nun tief verehrt wird, der Zirkusdirektor aber seine Gäste mit „hoch verehrtes Publikum“ anredet, darüber klärt uns die Sprache leider nicht auf. Positiv aber ist beides.  Ähnlich steht’s mit dem Gefühl, nur kann man hier die Stimmungen unterscheiden. Im Hochgefühl geht es um Freude, Jubel und Heiterkeit, im tiefen Gefühl eher um Nachdenkliches oder Trauriges.  Hochgefühl des Glücks und tiefe Dankbarkeit  Der Rennfahrer kann im Hochgefühl seines Erfolgs zufrieden in die Kamera lächeln. Der beinah in Seenot Ertrunkene empfindet ein tiefes Gefühl der Dankbarkeit für seine Retter. Im Hochgefühl des Glücks lädt der Lottogewinner alle seine Bekannten zu einer großen Party ein, mit einem tiefen Gefühl der Zuneigung blickt der Pfarrer auf seine Gemeinde. Und wer alle Höhen und Tiefen des Lebens und der Gefühle schon einmal durchgemacht hat, den kann man zu Recht als weise und erfahren bezeichnen.\newline\newline \newline\newline \section{Vocabular} abgrundtief unermesslich; sehr (meist negativ) \newline\newline  bis über beide Ohren verliebt umgangssprachlich für: sehr verliebt \newline\newline  bis über beide Ohren verliebt umgangssprachlich für: sehr verliebt \newline\newline  im siebten Himmel sein sehr glücklich sein \newline\newline  aus allen Wolken fallen umgangssprachlich für: völlig überrascht sein \newline\newline  bodenständig lange an einem Ort ansässig; in einer bestimmten Region verwurzelt; unkompliziert \newline\newline  weitschweifig sehr ausführlich,  umständlich, wortreich \newline\newline  hochfliegend sehr ambitioniert \newline\newline  Besser der Spatz in der Hand als die Taube auf dem Dach Redensart: Sei zufrieden mit dem, was du hast \newline\newline  Nacktschnecke, -n (f.) eine Schneckenart ohne Gehäuse \newline\newline  ein Silberstreif am Horizont umgangssprachlich für: etwas, das Hoffnung macht, ein schwacher Hoffnungsschimmer \newline\newline  die oberen Zehntausend umgangssprachlich für: sehr reiche Menschen \newline\newline  Seenot (f.) eine Situation, in der ein Mensch auf dem Meer in Lebensgefahr ist \newline\newline \end{document}