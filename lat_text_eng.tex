\documentclass[14pt,a4paper]{extarticle}
\usepackage[english]{babel}
\usepackage[letterpaper,top=1.5cm,bottom=1.5cm,left=1.5cm,right=1.5cm,marginparwidth=1.5cm]{geometry}
\begin{document}
\section{Text with Blanks} What Is Imposter Syndrome? \newline\newline 
Impostor syndrome (IS) refers (...) an internal experience (...) believing (...) you are not (...) competent (...) others perceive you (...) be. (...) (...) definition is usually narrowly applied (...) intelligence and achievement, (...) has links (...) perfectionism and (...) social context.
\newline\newline 
To put (...) simply, imposter syndrome is (...) experience (...) feeling (...) (...) phony—you feel (...) (...) (...) any moment you are going (...) be found (...) (...) (...) fraud—like you don't belong where you are, and you only got there through dumb luck. (...) can affect anyone no matter their social status, work background, skill level, or degree (...) expertise.
\newline\newline 
The term (...) (...) first used (...) psychologists Suzanna Imes and Pauline Rose Clance (...) (...) 1970s. (...) (...) concept (...) IS (...) introduced, (...) (...) originally thought (...) apply mostly (...) high-achieving women. Since then, (...) has (...) recognized (...) (...) widely experienced.
\newline\newline  Characteristics \newline\newline 
Some (...) (...) common signs (...) imposter syndrome include:
\newline\newline 
Self-doubtAn inability (...) realistically assess your competence and skillsAttributing your success (...) external factorsBerating your performanceFear (...) you won't live (...) (...) expectationsOverachieving Sabotaging your own successSetting very challenging goals and feeling disappointed (...) you fall short
\newline\newline 
While (...) some people, impostor syndrome can fuel feelings (...) motivation (...) achieve, (...) usually comes (...) (...) cost (...) (...) form (...) constant anxiety. You might over-prepare or work much harder than necessary (...) "make sure" (...) nobody finds (...) you are (...) fraud.
\newline\newline 
This sets (...) (...) vicious cycle, (...) which you think (...) (...) only reason you survived (...) class presentation (...) (...) you stayed (...) all night rehearsing. Or, you think (...) only reason you got through (...) party or family gathering (...) (...) you memorized details (...) all (...) guests so (...) you would always (...) ideas (...) small talk.
\newline\newline 
The problem (...) impostor syndrome is (...) (...) experience (...) doing well (...) something does nothing (...) change your beliefs. Even (...) you might sail through (...) performance or (...) lunch (...) coworkers, (...) thought still nags (...) your head, "What gives me (...) right (...) be here?" (...) (...) you accomplish, (...) (...) you just feel (...) (...) fraud. It's (...) (...) you can't internalize your experiences (...) success.
\newline\newline 
This makes sense (...) terms (...) social anxiety if you received early feedback (...) you were not good (...) social or performance situations. Your core beliefs (...) yourself are so strong, (...) they don't change, even (...) there is evidence (...) (...) contrary.
\newline\newline 
The thought process is: If you do well, (...) must be (...) result (...) luck because (...) socially incompetent person just doesn't belong.
\newline\newline 
Eventually, these feelings worsen anxiety and may lead (...) depression. People who experience impostor syndrome also tend not (...) talk (...) how they are feeling (...) anyone and struggle (...) silence, just (...) do those (...) social anxiety disorder.\newline\newline  Identifying \newline\newline 
While impostor syndrome is not (...) recognized disorder (...) (...) Diagnostic and Statistical Manual (...) Mental Disorders (DSM-5), (...) is not uncommon. (...) is estimated (...) 70% (...) people will experience (...) least one episode (...) (...) phenomenon (...) their lives.
\newline\newline 
If you think you might (...) imposter syndrome, ask yourself (...) following questions:
\newline\newline 
Do you agonize over even (...) smallest mistakes or flaws (...) your work?Do you attribute your success (...) luck or outside factors?Are you very sensitive (...) even constructive criticism?Do you feel (...) you will inevitably be found (...) (...) (...) phony?Do you downplay your own expertise, even (...) areas where you are genuinely (...) skilled than others?
\newline\newline 
If you often find yourself feeling (...) you are (...) fraud or an imposter, (...) may be helpful (...) talk (...) (...) therapist. (...) negative thinking, self-doubt, and self-sabotage (...) often characterize imposter syndrome can (...) an effect (...) many areas (...) your life.
\newline\newline  Causes \newline\newline 
We know (...) certain factors can contribute (...) (...) (...) general experience (...) impostor syndrome. (...) example, you might (...) come (...) (...) family (...) highly valued achievement or (...) parents who flipped back and forth (...) offering praise and being critical.
\newline\newline 
We also know (...) entering (...) new role can trigger impostor syndrome. (...) example, starting college or university might leave you feeling (...) (...) you don't belong and are not capable.
\newline\newline 
Impostor syndrome and social anxiety may overlap. (...) person (...) social anxiety disorder (SAD) may feel (...) (...) they don't belong (...) social or performance situations.
\newline\newline 
You might be (...) (...) conversation (...) someone and feel (...) (...) they are going (...) discover your social incompetence. You might be delivering (...) presentation and feel (...) (...) you just need (...) (...) through (...) before anyone realizes you really don't belong there.
\newline\newline 
While (...) symptoms (...) social anxiety can fuel feelings (...) imposter syndrome, (...) does not mean (...) everyone (...) imposter syndrome has social anxiety or vice versa. People without social anxiety can also feel (...) lack (...) confidence and competence. Imposter syndrome often causes normally non-anxious people (...) experience (...) sense (...) anxiety (...) they are (...) situations where they feel inadequate.
\newline\newline  Types \newline\newline 
Imposter syndrome can appear (...) (...) number (...) different ways. (...) few different types (...) imposter syndrome (...) (...) (...) identified are:
\newline\newline 
The perfectionist: Perfectionists are never satisfied and always feel (...) their work could be better. Rather than focus (...) their strengths, they tend (...) fixate (...) any flaws or mistakes. (...) often leads (...) (...) great deal (...) self-pressure and high amounts (...) anxiety.
The superhero: Because these individuals feel inadequate, they feel compelled (...) push themselves (...) work (...) hard (...) possible. 
The expert: These individuals are always trying (...) learn (...) and are never satisfied (...) their level (...) understanding. Even (...) they are often highly skilled, they underrate their own expertise.
The natural genius: These individuals set excessively lofty goals (...) themselves, and then feel crushed (...) they don't succeed (...) their first try.
The soloist: These people tend (...) be very individualistic and prefer (...) work alone. Self-worth often stems (...) their productivity, so they often reject offers (...) assistance. They tend (...) see asking (...) help (...) (...) sign (...) weakness or incompetence. 
\newline\newline  Coping \newline\newline 
To (...) past impostor syndrome, you need (...) start asking yourself some hard questions. They might include things such (...) (...) following:
\newline\newline 
"What core beliefs do I hold (...) myself?""Do I believe I am worthy (...) love (...) I am?""Must I be perfect (...) others (...) approve (...) me?"
\newline\newline \textbf{
Perfectionism plays (...) significant role (...) impostor syndrome. You might think (...) there is some perfect "script" (...) conversations and (...) you cannot say (...) wrong thing. You probably (...) trouble asking (...) help (...) others and may procrastinate due (...) your own high standards.}\newline\newline 
To move past these feelings, you need (...) become comfortable confronting some (...) those deeply ingrained beliefs you hold (...) yourself. (...) can be hard because you might not even realize (...) you hold them, but here are some techniques you can use:
\newline\newline 
Share your feelings. Talk (...) other people (...) how you are feeling. These irrational beliefs tend (...) fester (...) they are hidden and not talked about.
Focus (...) others. (...) (...) might feel counterintuitive, try (...) help others (...) (...) same situation (...) you. If you see someone who seems awkward or alone, ask (...) person (...) question (...) bring (...) into (...) group. (...) you practice your skills, you will build confidence (...) your own abilities.
Assess your abilities. If you (...) long-held beliefs (...) your incompetence (...) social and performance situations, make (...) realistic assessment (...) your abilities. Write down your accomplishments and what you are good at, and compare (...) (...) your self-assessment.
Take baby steps. Don't focus (...) doing things perfectly, but rather, do things reasonably well and reward yourself (...) taking action. (...) example, (...) (...) group conversation, offer an opinion or share (...) story (...) yourself.
Question your thoughts. (...) you start (...) assess your abilities and (...) baby steps, question (...) your thoughts are rational. Does (...) make sense (...) you are (...) fraud, given everything (...) you know?
Stop comparing. Every time you compare yourself (...) others (...) (...) social situation, you will find some fault (...) yourself (...) fuels (...) feeling (...) not being good enough or not belonging. Instead, during conversations, focus (...) listening (...) what (...) other person is saying. Be genuinely interested (...) learning more.
Use social media moderately. We know (...) (...) overuse (...) social media may be related (...) feelings (...) inferiority. If you try (...) portray an image (...) social media (...) doesn't match who you really are or (...) is impossible (...) achieve, (...) will only make your feelings (...) being (...) fraud worse.
Stop fighting your feelings. Don't fight (...) feelings (...) not belonging. Instead, try (...) lean into (...) and accept them. It's only (...) you acknowledge (...) (...) you can start (...) unravel those core beliefs (...) are holding you back.
Refuse (...) let (...) hold you back. No matter how much you feel (...) you don't belong, don't let (...) stop you (...) pursuing your goals. Keep going and refuse (...) be stopped.
\newline\newline  (...) Word (...) Verywell \newline\newline 
Remember (...) if you are feeling (...) an impostor, (...) means you (...) some degree (...) success (...) your life (...) you are attributing (...) luck. Try instead (...) turn (...) feeling into one (...) gratitude. Look (...) what you (...) accomplished (...) your life and be grateful.
\newline\newline 
Don't be crippled (...) your fear (...) being found out. Instead, lean into (...) feeling and (...) (...) (...) roots. Let your guard down and let others see (...) real you. If you've done all these things and still feel (...) your feeling (...) being an impostor is holding you back, (...) is important (...) speak (...) (...) mental health professional.
\newline\newline 
If you or (...) loved one are struggling (...) mental health, contact the Substance Abuse and Mental Health Services Administration (SAMHSA) National Helpline at 1-800-662-4357 (...) information (...) support and treatment facilities (...) your area.
For (...) mental health resources, see our National Helpline Database.\newline\newline \newline\newline \section{Text without Blanks} What Is Imposter Syndrome? \newline\newline 
Impostor syndrome (IS) refers to an internal experience of believing that you are not as competent as others perceive you to be. While this definition is usually narrowly applied to intelligence and achievement, it has links to perfectionism and the social context.
\newline\newline 
To put it simply, imposter syndrome is the experience of feeling like a phony—you feel as though at any moment you are going to be found out as a fraud—like you don't belong where you are, and you only got there through dumb luck. It can affect anyone no matter their social status, work background, skill level, or degree of expertise.
\newline\newline 
The term that was first used by psychologists Suzanna Imes and Pauline Rose Clance in the 1970s. When the concept of IS was introduced, it was originally thought to apply mostly to high-achieving women. Since then, it has been recognized as more widely experienced.
\newline\newline  Characteristics \newline\newline 
Some of the common signs of imposter syndrome include:
\newline\newline 
Self-doubtAn inability to realistically assess your competence and skillsAttributing your success to external factorsBerating your performanceFear that you won't live up to expectationsOverachieving Sabotaging your own successSetting very challenging goals and feeling disappointed when you fall short
\newline\newline 
While for some people, impostor syndrome can fuel feelings of motivation to achieve, this usually comes at a cost in the form of constant anxiety. You might over-prepare or work much harder than necessary to "make sure" that nobody finds out you are a fraud.
\newline\newline 
This sets up a vicious cycle, in which you think that the only reason you survived that class presentation was that you stayed up all night rehearsing. Or, you think the only reason you got through that party or family gathering was that you memorized details about all the guests so that you would always have ideas for small talk.
\newline\newline 
The problem with impostor syndrome is that the experience of doing well at something does nothing to change your beliefs. Even though you might sail through a performance or have lunch with coworkers, the thought still nags in your head, "What gives me the right to be here?" The more you accomplish, the more you just feel like a fraud. It's as though you can't internalize your experiences of success.
\newline\newline 
This makes sense in terms of social anxiety if you received early feedback that you were not good at social or performance situations. Your core beliefs about yourself are so strong, that they don't change, even when there is evidence to the contrary.
\newline\newline 
The thought process is: If you do well, it must be the result of luck because a socially incompetent person just doesn't belong.
\newline\newline 
Eventually, these feelings worsen anxiety and may lead to depression. People who experience impostor syndrome also tend not to talk about how they are feeling with anyone and struggle in silence, just as do those with social anxiety disorder.\newline\newline  Identifying \newline\newline 
While impostor syndrome is not a recognized disorder in the Diagnostic and Statistical Manual of Mental Disorders (DSM-5), it is not uncommon. It is estimated that 70% of people will experience at least one episode of this phenomenon in their lives.
\newline\newline 
If you think you might have imposter syndrome, ask yourself the following questions:
\newline\newline 
Do you agonize over even the smallest mistakes or flaws in your work?Do you attribute your success to luck or outside factors?Are you very sensitive to even constructive criticism?Do you feel like you will inevitably be found out as a phony?Do you downplay your own expertise, even in areas where you are genuinely more skilled than others?
\newline\newline 
If you often find yourself feeling like you are a fraud or an imposter, it may be helpful to talk to a therapist. The negative thinking, self-doubt, and self-sabotage that often characterize imposter syndrome can have an effect on many areas of your life.
\newline\newline  Causes \newline\newline 
We know that certain factors can contribute to the more general experience of impostor syndrome. For example, you might have come from a family that highly valued achievement or had parents who flipped back and forth between offering praise and being critical.
\newline\newline 
We also know that entering a new role can trigger impostor syndrome. For example, starting college or university might leave you feeling as though you don't belong and are not capable.
\newline\newline 
Impostor syndrome and social anxiety may overlap. A person with social anxiety disorder (SAD) may feel as though they don't belong in social or performance situations.
\newline\newline 
You might be in a conversation with someone and feel as though they are going to discover your social incompetence. You might be delivering a presentation and feel as though you just need to get through it before anyone realizes you really don't belong there.
\newline\newline 
While the symptoms of social anxiety can fuel feelings of imposter syndrome, this does not mean that everyone with imposter syndrome has social anxiety or vice versa. People without social anxiety can also feel a lack of confidence and competence. Imposter syndrome often causes normally non-anxious people to experience a sense of anxiety when they are in situations where they feel inadequate.
\newline\newline  Types \newline\newline 
Imposter syndrome can appear in a number of different ways. A few different types of imposter syndrome that have been identified are:
\newline\newline 
The perfectionist: Perfectionists are never satisfied and always feel that their work could be better. Rather than focus on their strengths, they tend to fixate on any flaws or mistakes. This often leads to a great deal of self-pressure and high amounts of anxiety.
The superhero: Because these individuals feel inadequate, they feel compelled to push themselves to work as hard as possible. 
The expert: These individuals are always trying to learn more and are never satisfied with their level of understanding. Even though they are often highly skilled, they underrate their own expertise.
The natural genius: These individuals set excessively lofty goals for themselves, and then feel crushed when they don't succeed on their first try.
The soloist: These people tend to be very individualistic and prefer to work alone. Self-worth often stems from their productivity, so they often reject offers of assistance. They tend to see asking for help as a sign of weakness or incompetence. 
\newline\newline  Coping \newline\newline 
To get past impostor syndrome, you need to start asking yourself some hard questions. They might include things such as the following:
\newline\newline 
"What core beliefs do I hold about myself?""Do I believe I am worthy of love as I am?""Must I be perfect for others to approve of me?"
\newline\newline \textbf{
Perfectionism plays a significant role in impostor syndrome. You might think that there is some perfect "script" for conversations and that you cannot say the wrong thing. You probably have trouble asking for help from others and may procrastinate due to your own high standards.}\newline\newline 
To move past these feelings, you need to become comfortable confronting some of those deeply ingrained beliefs you hold about yourself. This can be hard because you might not even realize that you hold them, but here are some techniques you can use:
\newline\newline 
Share your feelings. Talk to other people about how you are feeling. These irrational beliefs tend to fester when they are hidden and not talked about.
Focus on others. While this might feel counterintuitive, try to help others in the same situation as you. If you see someone who seems awkward or alone, ask that person a question to bring them into the group. As you practice your skills, you will build confidence in your own abilities.
Assess your abilities. If you have long-held beliefs about your incompetence in social and performance situations, make a realistic assessment of your abilities. Write down your accomplishments and what you are good at, and compare that with your self-assessment.
Take baby steps. Don't focus on doing things perfectly, but rather, do things reasonably well and reward yourself for taking action. For example, in a group conversation, offer an opinion or share a story about yourself.
Question your thoughts. As you start to assess your abilities and take baby steps, question whether your thoughts are rational. Does it make sense that you are a fraud, given everything that you know?
Stop comparing. Every time you compare yourself to others in a social situation, you will find some fault with yourself that fuels the feeling of not being good enough or not belonging. Instead, during conversations, focus on listening to what the other person is saying. Be genuinely interested in learning more.
Use social media moderately. We know that the overuse of social media may be related to feelings of inferiority. If you try to portray an image on social media that doesn't match who you really are or that is impossible to achieve, it will only make your feelings of being a fraud worse.
Stop fighting your feelings. Don't fight the feelings of not belonging. Instead, try to lean into them and accept them. It's only when you acknowledge them that you can start to unravel those core beliefs that are holding you back.
Refuse to let it hold you back. No matter how much you feel like you don't belong, don't let that stop you from pursuing your goals. Keep going and refuse to be stopped.
\newline\newline  A Word From Verywell \newline\newline 
Remember that if you are feeling like an impostor, it means you have some degree of success in your life that you are attributing to luck. Try instead to turn that feeling into one of gratitude. Look at what you have accomplished in your life and be grateful.
\newline\newline 
Don't be crippled by your fear of being found out. Instead, lean into that feeling and get at its roots. Let your guard down and let others see the real you. If you've done all these things and still feel like your feeling of being an impostor is holding you back, it is important to speak to a mental health professional.
\newline\newline 
If you or a loved one are struggling with mental health, contact the Substance Abuse and Mental Health Services Administration (SAMHSA) National Helpline at 1-800-662-4357 for information on support and treatment facilities in your area.
For more mental health resources, see our National Helpline Database.\newline\newline \newline\newline \end{document}